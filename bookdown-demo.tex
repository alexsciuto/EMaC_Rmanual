\documentclass[]{book}
\usepackage{lmodern}
\usepackage{amssymb,amsmath}
\usepackage{ifxetex,ifluatex}
\usepackage{fixltx2e} % provides \textsubscript
\ifnum 0\ifxetex 1\fi\ifluatex 1\fi=0 % if pdftex
  \usepackage[T1]{fontenc}
  \usepackage[utf8]{inputenc}
\else % if luatex or xelatex
  \ifxetex
    \usepackage{mathspec}
  \else
    \usepackage{fontspec}
  \fi
  \defaultfontfeatures{Ligatures=TeX,Scale=MatchLowercase}
\fi
% use upquote if available, for straight quotes in verbatim environments
\IfFileExists{upquote.sty}{\usepackage{upquote}}{}
% use microtype if available
\IfFileExists{microtype.sty}{%
\usepackage[]{microtype}
\UseMicrotypeSet[protrusion]{basicmath} % disable protrusion for tt fonts
}{}
\PassOptionsToPackage{hyphens}{url} % url is loaded by hyperref
\usepackage[unicode=true]{hyperref}
\hypersetup{
            pdftitle={The EMaC R Manual},
            pdfauthor={Alex Sciuto},
            pdfborder={0 0 0},
            breaklinks=true}
\urlstyle{same}  % don't use monospace font for urls
\usepackage{natbib}
\bibliographystyle{apalike}
\usepackage{color}
\usepackage{fancyvrb}
\newcommand{\VerbBar}{|}
\newcommand{\VERB}{\Verb[commandchars=\\\{\}]}
\DefineVerbatimEnvironment{Highlighting}{Verbatim}{commandchars=\\\{\}}
% Add ',fontsize=\small' for more characters per line
\usepackage{framed}
\definecolor{shadecolor}{RGB}{248,248,248}
\newenvironment{Shaded}{\begin{snugshade}}{\end{snugshade}}
\newcommand{\KeywordTok}[1]{\textcolor[rgb]{0.13,0.29,0.53}{\textbf{#1}}}
\newcommand{\DataTypeTok}[1]{\textcolor[rgb]{0.13,0.29,0.53}{#1}}
\newcommand{\DecValTok}[1]{\textcolor[rgb]{0.00,0.00,0.81}{#1}}
\newcommand{\BaseNTok}[1]{\textcolor[rgb]{0.00,0.00,0.81}{#1}}
\newcommand{\FloatTok}[1]{\textcolor[rgb]{0.00,0.00,0.81}{#1}}
\newcommand{\ConstantTok}[1]{\textcolor[rgb]{0.00,0.00,0.00}{#1}}
\newcommand{\CharTok}[1]{\textcolor[rgb]{0.31,0.60,0.02}{#1}}
\newcommand{\SpecialCharTok}[1]{\textcolor[rgb]{0.00,0.00,0.00}{#1}}
\newcommand{\StringTok}[1]{\textcolor[rgb]{0.31,0.60,0.02}{#1}}
\newcommand{\VerbatimStringTok}[1]{\textcolor[rgb]{0.31,0.60,0.02}{#1}}
\newcommand{\SpecialStringTok}[1]{\textcolor[rgb]{0.31,0.60,0.02}{#1}}
\newcommand{\ImportTok}[1]{#1}
\newcommand{\CommentTok}[1]{\textcolor[rgb]{0.56,0.35,0.01}{\textit{#1}}}
\newcommand{\DocumentationTok}[1]{\textcolor[rgb]{0.56,0.35,0.01}{\textbf{\textit{#1}}}}
\newcommand{\AnnotationTok}[1]{\textcolor[rgb]{0.56,0.35,0.01}{\textbf{\textit{#1}}}}
\newcommand{\CommentVarTok}[1]{\textcolor[rgb]{0.56,0.35,0.01}{\textbf{\textit{#1}}}}
\newcommand{\OtherTok}[1]{\textcolor[rgb]{0.56,0.35,0.01}{#1}}
\newcommand{\FunctionTok}[1]{\textcolor[rgb]{0.00,0.00,0.00}{#1}}
\newcommand{\VariableTok}[1]{\textcolor[rgb]{0.00,0.00,0.00}{#1}}
\newcommand{\ControlFlowTok}[1]{\textcolor[rgb]{0.13,0.29,0.53}{\textbf{#1}}}
\newcommand{\OperatorTok}[1]{\textcolor[rgb]{0.81,0.36,0.00}{\textbf{#1}}}
\newcommand{\BuiltInTok}[1]{#1}
\newcommand{\ExtensionTok}[1]{#1}
\newcommand{\PreprocessorTok}[1]{\textcolor[rgb]{0.56,0.35,0.01}{\textit{#1}}}
\newcommand{\AttributeTok}[1]{\textcolor[rgb]{0.77,0.63,0.00}{#1}}
\newcommand{\RegionMarkerTok}[1]{#1}
\newcommand{\InformationTok}[1]{\textcolor[rgb]{0.56,0.35,0.01}{\textbf{\textit{#1}}}}
\newcommand{\WarningTok}[1]{\textcolor[rgb]{0.56,0.35,0.01}{\textbf{\textit{#1}}}}
\newcommand{\AlertTok}[1]{\textcolor[rgb]{0.94,0.16,0.16}{#1}}
\newcommand{\ErrorTok}[1]{\textcolor[rgb]{0.64,0.00,0.00}{\textbf{#1}}}
\newcommand{\NormalTok}[1]{#1}
\usepackage{longtable,booktabs}
% Fix footnotes in tables (requires footnote package)
\IfFileExists{footnote.sty}{\usepackage{footnote}\makesavenoteenv{long table}}{}
\usepackage{graphicx,grffile}
\makeatletter
\def\maxwidth{\ifdim\Gin@nat@width>\linewidth\linewidth\else\Gin@nat@width\fi}
\def\maxheight{\ifdim\Gin@nat@height>\textheight\textheight\else\Gin@nat@height\fi}
\makeatother
% Scale images if necessary, so that they will not overflow the page
% margins by default, and it is still possible to overwrite the defaults
% using explicit options in \includegraphics[width, height, ...]{}
\setkeys{Gin}{width=\maxwidth,height=\maxheight,keepaspectratio}
\IfFileExists{parskip.sty}{%
\usepackage{parskip}
}{% else
\setlength{\parindent}{0pt}
\setlength{\parskip}{6pt plus 2pt minus 1pt}
}
\setlength{\emergencystretch}{3em}  % prevent overfull lines
\providecommand{\tightlist}{%
  \setlength{\itemsep}{0pt}\setlength{\parskip}{0pt}}
\setcounter{secnumdepth}{5}
% Redefines (sub)paragraphs to behave more like sections
\ifx\paragraph\undefined\else
\let\oldparagraph\paragraph
\renewcommand{\paragraph}[1]{\oldparagraph{#1}\mbox{}}
\fi
\ifx\subparagraph\undefined\else
\let\oldsubparagraph\subparagraph
\renewcommand{\subparagraph}[1]{\oldsubparagraph{#1}\mbox{}}
\fi

% set default figure placement to htbp
\makeatletter
\def\fps@figure{htbp}
\makeatother

\usepackage{booktabs}
\usepackage{amsthm}
\makeatletter
\def\thm@space@setup{%
  \thm@preskip=8pt plus 2pt minus 4pt
  \thm@postskip=\thm@preskip
}
\makeatother

\title{The EMaC R Manual}
\author{Alex Sciuto}
\date{Last Updated Febuary 2020}

\begin{document}
\maketitle

{
\setcounter{tocdepth}{1}
\tableofcontents
}
\chapter{How to Install R}\label{download}

This is the EMaC manual for how to wrangle, analyze, and visualize data
in \textbf{R}. I will review a few key packages, in addition to explaing
their key functions, with real data examples.

There are two things you need to do to install R on your computer.
First, you would need to install the latest version of R which you would
download and install from this link:
\href{https://cloud.r-project.org/}{R-download}. Once you run and
install R onto your computer, you would need to install R-Studio. R
studio is the graphic user interface (GUI) where you can place all of
your R-code. Follow this link:
\href{https://rstudio.com/products/rstudio/download/}{R-Studio Desktop
Download}. Once you have these two programs installed, all you need to
do is launch R-studio Desktop and you are ready to go.

\chapter{Introduction to Programming in R}\label{intro}

s

Before you can do data wrangling, statiscs, and visualization using R
for cognitive psychology research, you need to learn the basic syntax in
R. This chapter will introduce the basics.

\section{Variables in R}\label{variables-in-r}

R can manipulate and wrangle all kinds of data, these data could be
stored in a handful of variables that we can then work with. The first
one we will be talking about is numeric.

\subsection{Numeric}\label{numeric}

a numeric is a number (including decimals) that can be stored within a
variable. Here is an example:

\textbf{\emph{\texttt{x\ =\ 2}}}

Here, we assigned the numeric value \texttt{2} to the variable
\texttt{x}. Thus, if we were to do arithemtic, \texttt{x} will be
treated as \texttt{2}. Moreover, there are specific functions in R we
can use in order to do arithmetic with numeric variables. Here are a few
examples:

\begin{longtable}[]{@{}ccc@{}}
\toprule
Arithemtic & R-Input & R-Output\tabularnewline
\midrule
\endhead
Addition: & \texttt{x\ +\ 2} & \texttt{4}\tabularnewline
Substraction: & \texttt{x\ -\ 2} & \texttt{0}\tabularnewline
Division: & \texttt{x\ /\ 2} & \texttt{1}\tabularnewline
Multiplication: & \texttt{x\ *\ 2} & \texttt{4}\tabularnewline
Exponent: & \texttt{x\ \^{}\ 2} & \texttt{4}\tabularnewline
\bottomrule
\end{longtable}

\subsection{Character}\label{character}

A character is a combination of characters (either letters and/or
numbers) that can be stored within a variable. Moreover, it is very
important that you place the character between quotation marks. Here is
an example:

\textbf{\emph{\texttt{x\ =\ "Cognitive\ Psychology"}}}

Here, we assigned the character value \texttt{"Cognitive\ Psychology"}
to the variable \texttt{x}. There are ways to manipulate character type
variables, and also modify them. However, we will touch on that later.

\subsection{Logical}\label{logical}

A logical can only have two possible values, \texttt{TRUE} and
\texttt{FALSE} that can be stored within a variable. This is not
typically a varaible type that you will need to assign variable values
to. However, the logical type variable is extremely important becuase a
lot of functions in R return a logical variable. We will dive into some
of these functions

\subsection{Vector}\label{vector}

A vector is multiple variables stored into one data-set. Vectors can
contain any variable type: character, numeric, string, and logical. When
you are declaring a a vector type variable, you typically have to use
the concatenate function \texttt{c()}. Here is an example:

\texttt{x\ =\ c(1,2,3,4)}

\texttt{x\ =\ c("I",\ "like",\ "Cognitive",\ "Psychology")}

\texttt{x\ =\ c(TRUE,\ FALSE,\ FALSE,\ TRUE)}

To reference a vector varaible, you need to use \texttt{{[}{]}} and
inside the brackets, you have to specify the index of where the given
variable in the vector is. Here is an example

\begin{Shaded}
\begin{Highlighting}[]
\NormalTok{x =}\StringTok{ }\KeywordTok{c}\NormalTok{(}\StringTok{"10"}\NormalTok{, }\StringTok{"20"}\NormalTok{, }\StringTok{"30"}\NormalTok{, }\StringTok{"40"}\NormalTok{)}
\NormalTok{x[}\DecValTok{4}\NormalTok{]}
\end{Highlighting}
\end{Shaded}

\begin{verbatim}
## [1] "40"
\end{verbatim}

Here, we assigned the numbers: \texttt{10,\ 20,\ 30,\ 40} to the vector
\texttt{x}. \texttt{"Cognitive\ Psychology"} to the variable \texttt{x}.
There are ways to manipulate character type variables, and also modify
them. However, we will touch on that later.

\section{Logical Operators \&
if-Statements}\label{logical-operators-if-statements}

Now that you have a basic understanding of how variables work in R, the
next step is to learn how to use logical operators in order to specifiy
what are the specific conditions under which you want your variables to
be manipulated. The main way to do this in R is by using logical
operators.

\textbf{\emph{\texttt{x\ =\ 2}}}

Here, we assigned the numeric value \texttt{2} to the variable
\texttt{x}. Thus, if we were to apply logical operators to \texttt{x},
\texttt{x} will be treated as \texttt{2}. Here are examples of all of
the logical operators:

\begin{longtable}[]{@{}lccc@{}}
\toprule
Logical Operators & Description & R-Input & R-Output\tabularnewline
\midrule
\endhead
\textless{} & less than & x \textless{} 10 & TRUE\tabularnewline
\textless{}= & less than or equal to & x \textless{}= 2 &
TRUE\tabularnewline
\textgreater{} & greater than & x \textgreater{} 1 & TRUE\tabularnewline
\textgreater{}= & greater than or equal to & x \textgreater{}= 3 &
FALSE\tabularnewline
== & exactly equal to & x == 9 & FALSE\tabularnewline
!= & not equal to & x != 10 & FALSE\tabularnewline
!x & Not x & !x & FALSE\tabularnewline
x \textbar{}~y & x OR y & x == 10 \textbar{}~x != 10 &
TRUE\tabularnewline
x \& y & x AND y & x == 10 \&x != 10 & FALSE\tabularnewline
\bottomrule
\end{longtable}

Now that you have a basic understanding of logical operators, we can use
them in order to specify code to manipulate variables in a particular
way. For this example, we will be working with the function
\texttt{if\_else()}. However, one question you may have is what is a
function in R? I will describe the structure of functions, starting with
a simple one.

\begin{longtable}[]{@{}clccc@{}}
\toprule
\begin{minipage}[b]{0.08\columnwidth}\centering\strut
function: if you give an input, the function will do some computation
and return an output\strut
\end{minipage} & \begin{minipage}[b]{0.17\columnwidth}\raggedright\strut
arguments: this is the structure of how the input is recieved, every
function has preference\strut
\end{minipage} & \begin{minipage}[b]{0.19\columnwidth}\centering\strut
output: this is what the function returns\strut
\end{minipage} & \begin{minipage}[b]{0.06\columnwidth}\centering\strut
input\strut
\end{minipage}\tabularnewline
\midrule
\endhead
\begin{minipage}[t]{0.08\columnwidth}\centering\strut
max(): this function finds the maximum number in a numeric vector\strut
\end{minipage} & \begin{minipage}[t]{0.17\columnwidth}\raggedright\strut
(c(1,2,3,4)): parameters are the things you input into teh function to
get a specified output\strut
\end{minipage} & \begin{minipage}[t]{0.19\columnwidth}\centering\strut
less than\strut
\end{minipage} & \begin{minipage}[t]{0.06\columnwidth}\centering\strut
x \textless{} 10\strut
\end{minipage} & \begin{minipage}[t]{0.06\columnwidth}\centering\strut
TRUE\strut
\end{minipage}\tabularnewline
\bottomrule
\end{longtable}

s

```

\chapter{Literature}\label{literature}

Here is a review of existing methods.

\chapter{Methods}\label{methods}

We describe our methods in this chapter.

\chapter{Applications}\label{applications}

Some \emph{significant} applications are demonstrated in this chapter.

\section{Example one}\label{example-one}

\section{Example two}\label{example-two}

\chapter{Final Words}\label{final-words}

We have finished a nice book.

\bibliography{book.bib,packages.bib}

\end{document}
